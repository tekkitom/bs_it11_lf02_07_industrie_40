\documentclass[aspectratio=169]{beamer}
% \documentclass{beamer}
\setbeamertemplate{footline}{%
  \begin{beamercolorbox}[wd=\paperwidth,ht=2.25ex,dp=1ex]{date in
head/foot}%
    \hspace*{1ex} \insertframenumber{} / \inserttotalframenumber
  \end{beamercolorbox}}%
\setbeamertemplate{navigation symbols}{}

\mode<presentation>
{
   \useoutertheme{infolines}
%    \usetheme{Singapore}
  \usetheme{Boadilla}
%  \usetheme[secheader]{Boadilla}
    \useinnertheme[shadow]{rounded}
%   \subject{Informatik}
}

% \mode<handout>
% {
%   \usetheme{Montpellier}
%   \usecolortheme{dove}
% %   \usepackage{pgfpages}
%   \pgfpagesuselayout{4 on 1}[a4paper,landscape,border shrink=5mm]
% }


\usepackage[german]{babel}
\usepackage[utf8]{inputenc}
\usepackage{times}
\usepackage[T1]{fontenc}
\usepackage{eurosym}
\usepackage{graphicx}
\usepackage{ulem}
\usepackage{listings}
\usepackage{beamerfoils}
\lstset{numbers=left, numberstyle=\tiny, stepnumber=2, numbersep=5pt, language = html}
% \usepackage[markup=nocolor,deletedmarkup=xout]{changes}

\title{Industrie 4.0 I/O bei Mikrocontrollern}

\author{Thomas Maul}
\institute[BWS Hofheim]{Brühlwiesenschule, Hofheim}

% \date[09.11.22] % (optional)
% {09. November 22}

 \MyLogo{\includegraphics[height=1cm]{bwslogo_3.png}}
% \MyLogo{\includegraphics[height=1cm]{../../../bilder/bwslogo_3.png}}


\begin{document}

\begin{frame}
  \titlepage
 % \hyperlink{Teil_2}{\beamerbutton{Go part 2}}
\end{frame}
\begin{frame}
  \frametitle{Industrie - Entwicklung}
  \begin{itemize}
      \item Automation
      \item Automation (Level 2)
      \item Zugriff via Internet (aktuell)
  \end{itemize}
\end{frame}

\begin{frame}
  \frametitle{Vorteile und Nachteile}
    \begin{description} 
        \item[+] Wartung
        \item[+] Produktion anpassbar 
        \item[-] Zugriffsschutz 
    \end{description}
\end{frame}

\begin{frame}
  \frametitle{IT und Industrie 4.0}
  \begin{itemize}
    \item Daten aus unsicher Quelle an Maschine
    \item alte SPS häufig nicht tauschbar, kein TLS o.Ä möglich!
    \item Mehrstufige Firewall
  \end{itemize}
\end{frame}

\begin{frame}
  \frametitle{I/O für Mikrocontroller}
  \begin{itemize}
      \item LED braucht ca 1,8\,V, 10 mA
      \item Arduino liefert 5\,V, 20 mA\\ LED bei 5\,V $\rightarrow$ geht kaputt!
      \item Differenz mit Widerstand verheizen.
      \item $U = R * I$\\ $R = \frac{U}{I}$
  \end{itemize}
\end{frame}

\begin{frame}[fragile]
    \frametitle{Zugriff aus Arduino auf I/O via C (klassisch)}
    \begin{lstlisting}
int main(){
    DDRB |= (1 << PB5);// LED an Arduino, Pin 13
    while(1){
        if(PINB & (1 << PB5))
            PORTB &= ~(1 << PB5);
        else
            PORTB |= (1 << PB5);
    }
}    
    \end{lstlisting}
\end{frame}

\begin{frame}[fragile]
    \frametitle{Syntax Arduino und C}
    default= Pin ist Eingang\\
    Pin 0 und 1 (PORTD0, PORTD1) nicht benutzen!! (seriall port)
    \pause
\begin{description}
    \item[Pin auf Ausgang] DDRn - Data Direction Register n = Port 
    \item[setzen (high / ein)] \lstinline!PORTn |= (1 << PnX);!
    \item[löschen (low / aus)] \lstinline!PORTn &= ~(1 << PnX);!
    \item[test / Abfrage] \lstinline!if(PINn & (1 << PnX))!
\end{description}
Woher weiß ich welcher Arduino Pin welcher Port ist?\\
microchip xplained mini 328p - Datenblatt
\end{frame}
\end{document}

